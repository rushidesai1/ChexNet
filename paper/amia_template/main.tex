\documentclass{amia}
\usepackage{graphicx}
\usepackage[labelfont=bf]{caption}
\usepackage[superscript,nomove]{cite}
\usepackage{color}


\begin{document}


\title{Chest X-Ray (CXR) Disease Diagnosis with DenseNet}

\author{Doug Beatty$^{1}$, Filip Juristovski$^{2}$, Rushi Desai$^{3}$, Mohamed Abdelrazik$^{4}$}

\institutes{
    $^1$Georgia Institute of Technology, Atlanta, Georgia\\
}

\maketitle

\noindent{\bf Abstract}

\textit{Abstract text goes here, justified and in italics.  The abstract would normally be one paragraph long.  See Table 1. for appropriate abstract length by submission type.}

\section*{Introduction}
This template should be used as a starting point for AMIA submissions.
It is important to review the AMIA Call for Participation (CFP) where types of submissions considered and general requirements for each submission type are listed. All submissions must conform to the format and presentation requirements described in the CFP and at the submission site.


\section*{Approach/Metrics}
This sentence has one reference citation\cite{ref1}.

This sentence has two reference citations\cite{ref1,ref2}.

More text of an additional paragraph, with a figure reference (Figure ~\ref{fig1}) and a figure inside a Word text box below.  Figures need to be placed as close to the corresponding text as possible and not extend beyond one page.\\
\begin{figure}[h!]
\centering
\includegraphics[scale=1]{pics/figure1.png}
\caption{Total allergy alerts, overridden alerts, or drug order cancelled.}
\label{fig1}
\end{figure}

This is additional text added just to show the one-column formatting.  This is additional text added just to show the one-column formatting.  This is additional text added just to show the one-column formatting.  This is additional text added just to show the one-column formatting.  This is additional text added just to show the one-column formatting.  This is additional text added just to show the one-column formatting.  This is additional text added just to show the one-column formatting.

This paragraph contains a reference to a table just below (Table 1).  All tables need to be placed as close to the corresponding text as possible, But each individual table should be on one page and not extend to multiple pages unless labeled as ``€œContinued"€.

\begin{table}[h]
\centering
\caption{Submission type, abstract length, and page length maximum for AMIA submissions.}
  \begin{tabular}{|l|l|l|}
  \hline
    \textbf{Submission Type}    & \textbf{Abstract Length}  & \textbf{Page Length Maximum**} \\ \hline
    Paper  & 125-150 words  & Ten   \\ \hline
    Student Paper  & 125-150 words  & Ten \\ \hline
    Poster  &50-75 words*   & One \\ \hline
    Podium  Abstract & 50-75 words*  & Two \\ \hline
    Panel   &150-200 words  & Three \\ \hline
    System Demonstrations    &150-200 words  & One \\ \hline
  \end{tabular}
\end{table}
*: All podium abstract and poster submissions must have a brief (50-75 words) abstract. The abstract does NOT have to be part of the document, but must be entered on the submission website in the Abstract box in Step 2.

**: \textcolor{red}{If your submission is longer than what is specified below, it will be rejected without review.}

This is another paragraph.

\section*{Experimental Results}
Experimental results are described here.

Lorem ipsum dolor sit amet, consectetur adipiscing elit. Nulla malesuada tempus lacus. Phasellus vestibulum ut dolor ut vestibulum. Sed tincidunt libero nibh. Donec nec accumsan felis. Etiam consectetur metus sit amet pellentesque fermentum. Morbi sagittis velit quis justo faucibus, quis mollis ipsum vestibulum. Phasellus condimentum quam id nisl fringilla fermentum. Curabitur vitae augue ornare, aliquet odio sit amet, faucibus metus. Maecenas finibus eget magna lacinia efficitur.


\section*{Discussion}
Discussion is here.

Vivamus bibendum pharetra varius. Vivamus viverra nisl sed ex fringilla, in rutrum mauris dignissim. Pellentesque malesuada mattis velit id bibendum. Nulla maximus, justo eu elementum blandit, urna turpis posuere sapien, vel facilisis tellus nunc sit amet turpis. Proin tincidunt rhoncus turpis, et rhoncus diam luctus ut. Sed id leo varius, vehicula augue quis, dictum ligula. Praesent eget ex nec arcu bibendum viverra id ut mi. Proin lobortis tempus tellus ut pretium. Aliquam pretium sapien eget urna venenatis tincidunt.

\section*{Conclusion}
Your conclusion goes at the end, followed by References, which must follow the Vancouver Style (see: www.icmje.org/index.html).  References begin below with a header that is centered.  Only the first word of an article title is capitalized in the References.

\makeatletter
\renewcommand{\@biblabel}[1]{\hfill #1.}
\makeatother



\bibliographystyle{unsrt}
\begin{thebibliography}{1}
\setlength\itemsep{-0.1em}

\bibitem{ref1}
Siamak N. Nabili, M. (2019). Chest X-Ray Normal, Abnormal Views, and Interpretation. [online] eMedicineHealth.
\bibitem{ref2}
CheXpert: A Large Dataset of Chest X-Rays and Competition for Automated Chest X-Ray Interpretation. [Internet]. Stanfordmlgroup.github.io. 2019.
\bibitem{ref3}
FAN M, XU S. Massive medical image retrieval system based on Hadoop. Journal of Computer Applications. 2013;33(12):3345-3349.
\bibitem{ref4}
Huang G, Liu Z, van der Maaten L, Weinberger K. Densely Connected Convolutional Networks [Internet]. arXiv.org. 2019.
\bibitem{ref5}
Rajpurkar P, Irvin J, Zhu K, Yang B, Mehta H, Duan T et al. CheXNet: Radiologist-Level Pneumonia Detection on Chest X-Rays with Deep Learning [Internet]. arXiv.org. 2019.
\bibitem{ref6}
Liu H, Wang L, Nan Y, Jin F, Pu J. SDFN: Segmentation-based Deep Fusion Network for Thoracic Disease Classification in Chest X-ray Images [Internet]. arXiv.org. 2019.
\bibitem{ref7}
Wang X, Peng Y, Lu L, Lu Z, Bagheri M, Summers R. ChestX-Ray8: Hospital-Scale Chest X-Ray Database and Benchmarks on Weakly-Supervised Classification and Localization of Common Thorax Diseases. 2019.
\bibitem{ref8}
Yao L. Weakly supervised medical diagnosis and localization from multiple resolutions [Internet]. Arxiv.org. 2019.
\bibitem{ref9}
Guendel S, Grbic S, Georgescu B, Zhou K, Ritschl L, Meier A et al. Learning to recognize Abnormalities in Chest X-Rays with Location-Aware Dense Networks [Internet]. arXiv.org. 2019.
\bibitem{ref10}
Raoof S, Feigin D, Sung A, Raoof S, Irugulpati L, Rosenow E. Interpretation of Plain Chest Roentgenogram. 2019.
\bibitem{ref11}
Zhou B, Khosla A, Lapedriza A, Oliva A, Torralba A. Learning deep features for discriminative localization [Internet]. Arxiv.org. 2019.

\bibitem{ref98}
Pryor TA, Gardner RM, Clayton RD, Warner HR. The HELP system. J Med Sys. 1983;7:87-101.
\bibitem{ref99}
Gardner RM, Golubjatnikov OK, Laub RM, Jacobson JT, Evans RS. Computer-critiqued blood ordering using the HELP system. Comput Biomed Res 1990;23:514-28.



\end{thebibliography}

\end{document}
